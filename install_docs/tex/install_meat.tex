\label{s:install}
\section{Licensing}
The conditions on the use of McStas can be read in the files
\verb+LICENSE+ and \verb+LICENSE.LIB+ in the distribution. Essentially,
McStas may be used and modified freely, and copies of the McStas source code 
may be distributed to others. 
New or modified component and instrument files may be shared by 
the user community, and we shall be glad to include user contributions in the package.

\section{Installing support Apps}

To get a fully functional McStas installation, a few support applications are required. Essentially, you will need a C compiler, Perl and Perl-Tk, as well as a plotter such as Matlab, Scilab or PGPLOT (which we do not support but is still functional). These should better be installed before McStas.

\subsection{C compiler}
\label{s:Ccompiler}

The C compiler is used for building the instrument simulation executable, which does the hard job. It is called transparently by the mcgui and mcrun McStas tools.

\begin{itemize}
\item{Win32: Bloodshed Dev-C++ (Win32).
  \label{s:instblood}
  To install Bloodshed Dev-C++, download the installer package from
  \\\
  \\\htmladdnormallink{http://www.bloodshed.net/dev/devcpp.html}{http://www.bloodshed.net/dev/devcpp.html}.\\\
  \\
  When installed, add the \texttt{C:$\backslash$Dev-Cpp$\backslash$bin} directory to your PATH
  using\\\ \\
  'Start/Settings/Control Panel/System/Advanced/Environment Variables'.
}
\item{Unix/Linux: standard C compiler.
  Most Unix/Linux systems come with a system C compiler, usually named 'cc' or 'gcc'. In case this would not be the case, install such a compiler from Linux packages (RedHat, SuSe or Debian), or compile it from scratch. 
}
\end{itemize}

\subsection{Gui tools (Perl + Tk)}
\label{s:perltk}

The McStas tools are written using the Perl language, with calls to the perl-Tk library.

\begin{itemize}
\item{Win32:}
  \begin{itemize}
  \item{Get and install the ActivePerl package from\\
      \htmladdnormallink{http://www.activestate.com/Products/Download/Register.plex?id=ActivePerl}{http://www.activestate.com/Products/Download/Register.plex?id=ActivePerl}
      \\(Registration
      not required)}
  \item{Get and install the ActiveTcl package from\\
      \htmladdnormallink{http://www.activestate.com/Products/Download/Register.plex?id=ActiveTcl}{http://www.activestate.com/Products/Download/Register.plex?id=ActiveTcl}
      \\(Registration
      not required)}
  \item{You may need to update/install the Microsoft Windows Installer. Get it from Windows Support Downloads at \htmladdnormallink{http://www.microsoft.com/}{http://www.microsoft.com/}.}
  \item{Make sure that during installation, you have requested to attach the .pl and .tcl extensions to Perl and Tcl, and the C:$\backslash$Perl$\backslash$bin is in the PATH.}
  \end{itemize}
\item{Unix/Linux:}
  \begin{itemize}
    \item{Install Perl, Tcl/Tk and perl-Tk. Prebuilt packages exist for
      most Linux distributions and also most other Unix-like operating
      systems. You will find these packages at \htmladdnormallink{http://freshmeat.net/}{http://freshmeat.net/}, \htmladdnormallink{http://www.rpmfind.net/linux/RPM/}{http://www.rpmfind.net/linux/RPM/} (for SuSe and RedHat) and using Fink Commander (for Mac OS X). Some Linux like system do not provide perl-Tk as an RPM/Debian file. You will then have to compile it from the source code (with ./configure; make; make install).}
    \item{Consult the McStas webpage at \htmladdnormallink{http://mcstas.risoe.dk}{http://mcstas.risoe.dk} for updated links to the source code distributions.}
    \end{itemize}
  \end{itemize}
  
\subsection{Plotting backends (All platforms)}
\label{s:plotting}
For plotting with McStas, different support packages can be used:
\begin{itemize}
\item{\htmladdnormallink{PGPLOT}{http://www.astro.caltech.edu/~tjp/pgplot/}/\htmladdnormallink{PDL (perl-PDL)}{http://pdl.perl.org}/\htmladdnormallink{pgperl (perl-PGPLOT)}{http://www.ast.cam.ac.uk/AAO/local/www/kgb/pgperl/} (Unix only) - Binary builds of the packages
    exist for various Linux distributions (for instance
    \htmladdnormallink{Debian}{http://www.debian.org} comes with
    prebuilt versions). Prebuilt versions also exist for some commercial Unix'es. 
    Refer to distributor/vendor for documentation. The packages can also be
    built from source using some (in many cases much) effort.  See the
    PGPLOT documentation for further details. 
    You may need to define the \verb+PGPLOT_DIR+ variable to the location of the Pgplot library. }
\item{Matlab (Some Unix/Win32) - refer to
    \htmladdnormallink{http://www.mathworks.com}{http://www.mathworks.com}. Matlab licenses are rather costly, but discount programmes for university and research departments exist.}
\item{Scilab (Unix/Win32/Mac...) - a free 'Matlab-like' package, available from
    \htmladdnormallink{http://www-rocq.inria.fr/scilab/}{http://www-rocq.inria.fr/scilab/}. McStas also requires the Plotlib library from \\ \htmladdnormallink{http://www.dma.utc.fr/$\sim$mottelet/myplot.html}{http://www.dma.utc.fr/~mottelet/myplot.html}. This package is now included in McStas and needs not be installed by the user.
    \\
On Windows systems, Scilab should be installed in a directory such as \texttt{C:$\backslash$Scilab} (do NOT use spaces in names) and you should then add the \texttt{C:$\backslash$Scilab$\backslash$bin} directory to your PATH using\\\ \\
'Start/Settings/Control Panel/System/Advanced/Environment Variables'.}
\end{itemize}

\section{Getting McStas}
\label{s:obtain}
The \htmladdnormallink{McStas}{http://mcstas.risoe.dk} package is
available in three different distribution packages, from the project
website at
\htmladdnormallink{http://mcstas.risoe.dk}{http://mcstas.risoe.dk}, e.g.
\begin{itemize}
\item{\texttt{mcstas-1.8-src.tar.gz}\\Source code package for
    building \htmladdnormallink{McStas}{http://mcstas.risoe.dk} on
    (at least) Linux and Windows 2000. This package should compile on
    most Unix platforms with an \texttt{ANSI-c} compiler. - Refer to section \ref{s:src}}
\item{\texttt{mcstas-1.8-i686-unknown-Linux.tar.gz}\\Binary package
  for Linux systems, currently built on Debian GNU/Linux 3.0 'woody'. 
  Should work on most Linux setups.
 - Refer to section \ref{s:linbin}}
\item{\texttt{mcstas-1.8-i686-unknown-Win32.zip}\\Binary package
  for Win32 systems, currently built on Microsoft Windows 2000
  professional, using the \texttt{gcc} 2.95 compiler from 
  \htmladdnormallink{Bloodshed Dev-C++ 5 Beta
    7}{http://www.bloodshed.net/dev/devcpp.html}
   - Refer to section \ref{s:winbin}}
\end{itemize}

\section{Source code build}
\label{s:src}
The \htmladdnormallink{McStas}{http://mcstas.risoe.dk} package is
beeing co-developed for mainly Linux and Windows systems, however 
the Linux build instructions below will work on most Unix
systems, including Mac OS X). For an updated list of platforms on which McStas has been
built, refer to the \htmladdnormallink{project website}{http://mcstas.risoe.dk/developments/platforms/}.

\subsection{Windows build}
\begin{itemize}
\item{Start by unpacking the \texttt{mcstas-1.8-src.tar.gz} package using
e.g. \htmladdnormallink{Winzip}{http://www.winzip.com}.}
\item{Compile the \htmladdnormallink{McStas}{http://mcstas.risoe.dk} package using the \texttt{build.bat} script of the
\texttt{mcstas-1.8} directory you just unpacked. Follow the on screen
instructions.} 
\item{When the build has been done (e.g. \texttt{mcstas.exe}
has been produced), proceed to install (Section \ref{s:winbin}).}
\end{itemize}

\subsection{Unix build}

McStas uses autoconf to detect the system configuration and creates the
proper Makefiles needed for compilation. On Unix-like systems, you
should be able to compile and install McStas using the following steps:
\begin{enumerate}
\item{Unpack the sources to somewhere convenient and change to the
    source directory:\\
  gunzip -c mcstas-1.8-src.tar.gz | tar xf -\\
  cd mcstas-1.8/}
\item{Configure and compile McStas:\\
  ./configure\\
  make}
\item{Install McStas (as superuser):\\
  make install}
\end{enumerate}

\noindent The installation of McStas in step 3 by default installs in the
/usr/local/ directory, which on most systems requires superuser (root)
privileges. To install in another directory, use the --prefix= option to
configure in step 2. For example,\\

  ./configure --prefix=/home/joe\\

\noindent will install the McStas programs in /home/joe/bin/ and the
library files needed by McStas in /home/joe/lib/mcstas/.

In case ./configure makes an incorrect guess, some environment variables
can be set to override the defaults:
\begin{itemize}
\item{The CC environment variable may be set to the name of the C compiler
   to use (this must be an ANSI C compiler). This will also be used for
   the automatic compilation of McStas simulations in mcgui and
   mcrun.}
\item{CFLAGS may be set to any options needed by the compiler (eg. for
   optimization or ANSI C conformance). Also used by mcgui/mcrun.}
\item{PERL may be set to the path of the Perl interpreter to use.}
\end{itemize}
\noindent To use these options, set the variables before running ./configure. Eg.\\

    setenv PERL /pub/bin/perl5

    ./configure\\

\noindent It may be necessary to remove configure's cache of old
choices first:\\


    rm -f config.cache\\


\noindent If you experience any problems, or have some questions or ideas
concerning McStas, please contact
\htmladdnormallink{peter.willendup@risoe.dk}{mailto:peter.willendup@risoe.dk}
or the McStas mailing list at \htmladdnormallink{neutron-mc@risoe.dk}{mailto:peter.willendup@risoe.dk}.

You should try to make sure that the directory containing the McStas
binaries (mcstas, gscan, mcdisplay, etc.) is contained in the PATH
environment variable. The default directory is /usr/local/bin, which is
usually, but not always, included in PATH. Alternatively, you can
reference the McStas programs using the full path name, ie.\\

  \verb+/usr/local/bin/mcstas my.instr+

  \verb+perl /usr/local/bin/mcrun -N10 -n1e5 mysim -f output ARG=42+

  \verb+perl /usr/local/bin/mcdisplay --multi mysim ARG=42+\\

This may also be necessary for the front-end programs if the install
procedure could not determine the location of the perl interpreter on
your system.

If McStas is installed properly, it should be able to find the files it
needs automatically. If not, you should set the MCSTAS environment
variable to the directory containing the runtime files "mcstas-r.c" and
"mcstas-r.h" and the standard components (*.comp). Use one of\\

  \verb+MCSTAS=/usr/local/lib/mcstas; export MCSTAS # sh, bash+

  \verb+setenv MCSTAS /usr/local/lib/mcstas         # csh, tcsh+\\

%The PGPLOT library, which can be used by the mcdisplay frontend
%(alternatives are Scilab or Matlab), needs the PGPLOT\_DIR environment
%variable to be set to the directory containing PGPLOT, eg.\\
%
%  \verb+PGPLOT_DIR=/usr/lib/pgplot; export PGPLOT_DIR # sh, bash+
%
%  \verb+setenv PGPLOT_DIR /usr/lib/pgplot             # csh, tcsh+\\
%
%\noindent See the PGPLOT documentation for further details.

\section{Binary install, Linux}
\label{s:linbin}
Should be very easy, simply start from 'make install' in Section \ref{s:src}.

\section{Binary install, Windows}
\label{s:winbin}
\begin{itemize}
\item{Start by unpacking the \texttt{mcstas-1.8-i686-unknown-Win32.zip} package using
e.g. \htmladdnormallink{Winzip}{http://www.winzip.com}.}
\item{Execute the \texttt{install.bat} installation script. Follow the
  on screen instructions.}
\item{Set the required (see output of \texttt{install.bat}) environment variables using
\\\ \\
\noindent 'Start/Settings/Control Panel/System/Advanced/Environment
Variables'\\\ 
  \begin{itemize}
    \item{\verb+PATH+ Append e.g. C:$\backslash$mcstas$\backslash$bin}
    \item{\verb+MCSTAS+ Create it as e.g. C:$\backslash$mcstas$\backslash$lib}
  \end{itemize}}
\end{itemize}
It is important that Perl is correctly installed to execute all the McStas tools (e.g. mcdoc.pl, mcrun.pl, mcgui.pl, \ldots). Create a shortcut of the \verb+C:\mcstas\bin\mcgui.pl+ on your Desktop (the icon should be a yellow dot). Whenever launched from the Windows Command window (cmd), you must specify the .pl extension to all McStas Perl script commands (e.g. 'mcrun.pl' and 'mcgui.pl', not 'mcrun' or 'mcgui') except for \verb+mcstas+ itself.

Actually, the modifications to your environnement variables should be
\begin{itemize}
    \item{\verb+PATH+ Append e.g. 
    C:$\backslash$mcstas$\backslash$bin;c:$\backslash$Program Files$\backslash$Scilab-2.7.2$\backslash$bin;C:$\backslash$Perl$\backslash$bin$\backslash$;C:$\backslash$Dev-Cpp$\backslash$bin;}
    \item{\verb+MCSTAS+ Create it as e.g. C:$\backslash$mcstas$\backslash$lib}
\end{itemize}
using menu item 'Start/Settings/Control Panel/System/Advanced/Environment Variables'.

\section{Testing the McStas distribution}
\label{s:testing}

The \verb+examples+ directory of the distribution contains a set of instrument examples. These are used for the McStas self test procedure, which is executed with
\begin{verbatim}
    mcrun --test
\end{verbatim}
This test takes a few minutes to complete, and ends with a short report on the installation itself, the simulation accuracy and the plotter check. 

You should now be able to use McStas. For some examples to try, see the
examples/ directory. Start 'mcgui' (mcgui.pl on Windows), and select one of the examples in the 'Neutron Sites' menu.

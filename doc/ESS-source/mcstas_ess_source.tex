\documentclass{elsarticle}
%\documentclass[12pt]{iopart}
%%%\bibliographystyle{iopart-num}
%\usepackage{citesort}
%\usepackage[LGR,T1]{fontenc}
\usepackage{textcomp} 
%\usepackage[greek]{babel}
\newcommand{\BibTeX}{Bib\TeX}
\newcommand{\REVTeX}{REV\TeX}

\sloppy 

\newcommand{\order}[1]{\ensuremath{\mathcal{O}\left({#1}\right)}}
\newcommand{\p}[1]{\ensuremath{\left ( #1 \right )}}
\newcommand{\noteesben}[1]{{\tiny\color{blue} [#1]}}
\usepackage{epsfig}
\usepackage{graphicx}
\usepackage[dvips]{color}
\usepackage{verbatim}
\usepackage{gensymb}
\usepackage{gensymb}
\usepackage{lineno}
%\usepackage[firstpage]{draftwatermark}
%\usepackage{draftwatermark}
%\usepackage[firstpage]{draftwatermark}
%\SetWatermarkScale{4}
%\usepackage{
%amssymb,amsfonts,amsmath,epsfig,
%mathrsfs,fancyhdr,
%a4wide,subfigure,graphics,
%graphicx,array,textcomp
%%,floatflt                                                                      
%,lscape,xspace, wasysym}
\def\blfootnote{\xdef\@thefnmark{}\@footnotetext} 
\long\def\symbolfootnote[#1]#2{\begingroup%
\def\thefootnote{\fnsymbol{footnote}}\footnote[#1]{#2}\endgroup} 

\begin{document}
\setpagewiselinenumbers
%\modulolinenumbers[5]
%\linenumbers
%\begin{center}
%\end{bf}
\begin{frontmatter}
\title{McStas ESS source}
% \title{This is a specimen title\tnoteref{t1,t2}}
%%%%\noindent \qquad \\[-6pt] \qquad  Version 1.1\\\qquad 
%\newline
%\today 
%\vspace{5mm}
\author{Peter Willendrup$^{1,2}$, Esben Klinkby$^{3,4}$}
\address{1) DTU Physics, Technical University of Denmark, DTU Lyngby Campus, Anker Engelunds Vej 1, DK-2800 Kgs. Lyngby, Denmark}
\address{2) ESS design update programme - Denmark}
\address{3) European Spallation Source ESS AB, Box 176, S-221 00 Lund, Sweden}
\address{4) DTU Nutech, Technical University of Denmark, DTU Ris\o~Campus,\\ Frederiksborgvej 399, DK-4000 Roskilde, Denmark}
\end{frontmatter}
\section{Abstract}
short description




\emph{Keywords}: Neutron, ESS, spallation, MCNPX, McStas, simulation, source, spectrum

\section{Introduction}
In this document the development of the ESS source description in McStas\cite{lefm:1999,will:2013,man,comp} is outlined by comparing brilliance curves of the various versions.

\section{Comparison between ESS source in McStas versions 1.12c, 2.0 and 2.0$+$}
\label{sec:val}

\begin{figure}[h!]
\begin{minipage}{\linewidth}
\centering
\epsfig{figure=figs/1.12c_cold_16.67Hz/Mean_brill.sim.ps,width=0.35\linewidth,angle=-90}
\epsfig{figure=figs/1.12c_cold_16.67Hz/Peak_brill.sim.ps,width=0.35\linewidth,angle=-90}
\epsfig{figure=figs/1.12c_cold_14Hz/Mean_brill.sim.ps,width=0.35\linewidth,angle=-90}
\epsfig{figure=figs/1.12c_cold_14Hz/Peak_brill.sim.ps,width=0.35\linewidth,angle=-90}
\caption{Average(left) and peak(right) cold brilliance curves obtained from running McStas 1.12c using frequenzy 16.67~Hz 3ms (above) and 14~Hz, 2.86~ms below}
\label{fig:2001}
\end{minipage}\hfill
\end{figure}


\begin{figure}[h!]
\begin{minipage}{\linewidth}
\centering
\epsfig{figure=figs/1.12c_therm_16.67Hz/Mean_brill.sim.ps,width=0.35\linewidth,angle=-90}
\epsfig{figure=figs/1.12c_therm_16.67Hz/Peak_brill.sim.ps,width=0.35\linewidth,angle=-90}
\epsfig{figure=figs/1.12c_therm_14Hz/Mean_brill.sim.ps,width=0.35\linewidth,angle=-90}
\epsfig{figure=figs/1.12c_therm_14Hz/Peak_brill.sim.ps,width=0.35\linewidth,angle=-90}
\caption{Average(left) and peak(right) thermal brilliance curves obtained from running McStas 1.12c using frequenzy 16.67~Hz 3ms (above) and 14~Hz, 2.86~ms below}
\label{fig:2001_t}
\end{minipage}\hfill
\end{figure}


Figures~\ref{fig:2001} and \ref{fig:2001_t} show the cold and thermal average and peak brightness respectively.

Figure~\ref{fig:2012vs2001} shows the average(left) and peak(right) brilliance curves obtained from running McStas 2.0 using 2012(upper) and 2001(lower) source descriptions respectively.








%measure flux / heat load /(spectra). Monitor flux/spectra in the beam extraction windows





%\section{References}

\bibliographystyle{elsarticle-num}
\bibliography{mybib}

%\newpage
%\section{Appendix}
%\begin{figure}[h!]
%\begin{minipage}{\linewidth}
%\centering
%\epsfig{figure=figs/ucn_impact_vs_flux_cold.eps,width=0.90\linewidth} %flux_vs_impact.eps
%\caption{Relation between cold flux (0-5~meV) available the lower cold/thermal beam-lines versus the flux available for UCN, central% in the through-going tube.}
%\label{fig:impact_vs_flux_cold}
%\end{minipage}\hfill
%\end{figure}
%\begin{figure}[h!]
%\begin{minipage}{\linewidth}
%\centering
%\epsfig{figure=figs/ucn_impact_vs_flux_interm.eps,width=0.90\linewidth} %flux_vs_impact.eps
%\caption{Relation between intermediate flux (5-20~meV) available the lower cold/thermal beam-lines versus the flux available for UCN%, central in the through-going tube.}
%\label{fig:impact_vs_flux_interm}
%\end{minipage}\hfill
%\end{figure}
%\begin{figure}
%\begin{minipage}{\linewidth}
%\centering
%\epsfig{figure=figs/ucn_impact_vs_flux_thermal.eps,width=0.90\linewidth} %flux_vs_impact.eps
%\caption{Relation between cold flux (20-100~meV) available the lower cold/thermal beam-lines versus the flux available for UCN, cent%ral in the through-going tube.}
%\label{fig:impact_vs_flux_thermal}
%\end{minipage}\hfill
%\end{figure}

\end{document}

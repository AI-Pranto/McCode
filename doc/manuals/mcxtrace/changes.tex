% Emacs settings: -*-mode: latex; TeX-master: "manual.tex"; -*-

\chapter{New features in \MCX \version}
\label{c:changes}

\MCX is an ongoing evolving project with features being added frequently. While we strive to test it's features thoroughly, bugs
are inevitable. Bugs are generally reported using the user-mailing list: mcxtrace-users@mcxtrace.org and subsequently (after initial triage)
tracked using the \MCX\ Trac system \cite{mccode_trac_webpage}
(shared with its sister project \MCS). We will not present here an extensive
list of improvements and corrections, and we let the reader refer to this bug reporting service
for details. Only important changes are indicated here.
Of course, we can not guarantee that the software is bullet proof, but we do our best to correct bugs, when they are reported.\index{Bugs}

%\section{General}
%\label{s:new-features:general}
%\begin{itemize}
%\end{itemize}

\section{Kernel}
\label{s:new-features:kernel}
\index{Kernel}

The following changes concern the 'Kernel' (i.e. the \MCX\ meta-language and program). See the dedicated chapter in the \textit{User manual}(This document) for more details.

\section{Run-time}
\label{s:new-features:run-time}
\index{Library!Run-time}
New intersection routines include \verb+ellipsoid_intersect+ and \verb+sphere_intersect+.


\section{Components and Library}
\label{s:new-features:components}
\index{Components} \index{Library!Components}
We here list the new and updated components (found in the \MCX \verb+lib+ directory)
which are detailed in the \textit{Component manual}. For an overview see the \textit{Component Overview} of the \textit{User Manual}(This Document).
%\subsection{General}
\subsection{New components}

\subsubsection*{Sources}
We have revised the following source models to make them work as conherently as possible while retaining the backwards compatibility.
\begin{description}
\item[Source\_pt] Point source
\item[Source\_flat] Flat surface uniform source
\item[Source\_div] Flat surface uniform source with divergence distribution
\item[Source\_gaussian] Gaussian crossection source approximating a Bending Magnet
\end{description}
Furthermore we now also supply an experimental model of a bending magnet \texttt{Bending\_magnet}.
The following experimental components have been set up to facilitate interfacing with SPECTRA, Simplex, and GENESIS 1.3.
In general they take the output of the relevant external program to get a phase space distribution of photons from which McXtrace samples rays
to trace.
\begin{description}
\item[Source\_genesis] Reads the output of a GENESIS 1.3 simulation. Useful for FEL-simulations. 
\item[Source\_spectra] Reads the output of a SPECTRA simulation. Useful for Synchrotron simulations.
\item[Source\_simplex] Reads the output of a Simplex simulation. Useful for FEL simulations.
\end{description}

\subsubsection*{Samples}
\begin{description}
\end{description}
\subsubsection*{Optics}
\begin{description}
\item[Multilayer\_elliptic] Elliptical multilayer mirror. This component now has the option of using a analytical kinematical approximation to compute refleticity as opposed to supplying a datafile. This is faster, but often less accurate.
\item[Mirror\_elliptic] Elliptical shape mirror. Major geometry bug-fixes. Is scheduled to be merged with \texttt{Mirror\_curved}
\item[Mirror\_parabolic] Parabolic shape mirror. Major geometry bug-fixes. Is scheduled to be merged with \texttt{Mirror\_curved}
\item[Mirror\_curved] Cylindrical mirror. Fixed a bug which prevented its use as an azimuthal focusing mirror, with sideways incidence. 


\end{description}
\subsubsection*{Miscellaneous}
\begin{description}
\item[Arm] Coordinate transformation
\item[Progress\_bar] Display progress of simulation
\end{description}
\subsubsection*{Monitors}
\begin{description}
\item[E\_monitor] Int. vs. X-ray energy
\item[L\_monitor] Int. vs. X-ray wavelength
\item[PSD\_monitor] Int. spatial coordinates
\item[PSD\_monitor\_4P] Int. vs. latitude and longitude
\end{description}


\subsection{New example instruments}


\begin{itemize}
\item \verb+SAXSLAB_SAXS.instr+ The SAXLAB (formerly JJ-Xray Systems) saxs-machine situated at Faculty of Life-Sciences, Univeristy of Copenhagen. 
\item \verb+ESRF_ID11.instr+ Model of the 3DXRD-microscope at ID11, ESRF
\item \verb+Be_BM.instr+ A model beamline to explore the idea of a "pink"-beam beamline using CRLs\cite{vaughan2011x}.
\end{itemize}

%\section{Documentation}
%\label{s:new-features:documentation}

\section{Tools, installation}
\label{s:new-features:tools}
\index{Tools}
\index{Installing}
\subsection{Selected Tool features}
\begin{itemize}
  \item standard FreeBSD port
    %  \item Support for per-user mcxtrace\_config.perl file, located in \verb+$HOME/.mcstas/+ . This folder is also the default
%     location of the 'host list' for use with MPI or gridding, simply name the file 'hosts'.
%  \item mxgui Save Configuration for saving chosen settings on the 'Configuration options' and 'Run dialogue'.
  \item Possibility to run MPI or grid simulations by default from mxgui.
%  \item When scanning parameters, mxrun now terminates with a relevant error message if one or more scan steps
%     failed (intensities explicitly set to 0 in those cases).
%  \item When running parameter optimisations, a logfile (default name is "mcoptim\_XXXX.dat" where XXXX is a
%     pseudo-random string) is created during the optimisation, updated at each optim step.
  \item We provide syntax-highlighting setup files for eamcs, vim and gedit editors.
\end{itemize}
\subsection{Platform support}
\begin{itemize}
\item Mac OS X 10.8 Mountain Lion, 10.9 Mavericks.
\item Windows 8, Windows 7, Windows XP.
\item "Any" Linux - reference platform is Debian 7.0.
\item BSD (FreeBSD release 10.0 tested)
\item Oracle OpenSolaris 10 (Intel tested, Sparc probably OK)
\item Plus probably any UNIX/POSIX type environment with a bit of effort...
\end{itemize}
Details about the installation and the available tools are given in chapter \ref{installing}.

%\subsection{Various}

\subsection{Warnings}
{\bfseries WARNING:} The 'dash' shell which is used as /bin/sh on some Linux system (Including Ubuntu 7.04) makes the 'Cancel' and 'Update' 
buttons fail in mcgui. Solutions are:
\begin{itemize}
\item[a)] If your system is a Debian or Ubuntu, please dpkg-reconfigure dash and say 'no' to install dash as /bin/sh
\item[b)] If you run another Linux with /bin/sh beeing dash, please install bash and manually change the /bin/sh link to point at bash.
\end{itemize}

%\section{Future extensions}
%\label{s:future}
%The following features are planned for the oncoming releases of \MCX\
%(not an ordered list):
%\begin{itemize}
%\item Increased validation and testing.
%\item Extend test cases to all (most) components. One instrument
%  pr. component. (Probably not in \verb+examples/+.
%\item Updates to mcresplot to support the Matlab and Scilab backends.
%\item Global changes of components relating to polarisation
%  visualisation.
%\item Visualisation of neutron spins in magnetic fields for all
%  graphical backends.
%\item \emph{Array} \verb+AT+ specifiers for components, i.e. \\
%  \verb+COMPONENT MyComp=Comp(...)+\\\verb+AT([Xarray],[Yarray],[Zarray])+ and\\
%  \verb+AT Positions('filename')+
%\item Gui support for array AT specifiers.
%\item More complete polarisation support including numerically defined
%  magnetic fields and advanced sample components.
%\item Perl or python plotting alternative to PGPLOT.
%\item Larger variety of sample components.
%\end{itemize}









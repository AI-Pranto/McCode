% Emacs settings: -*-mode: latex; TeX-master: "manual.tex"; -*-

\addcontentsline{toc}{chapter}{\protect\numberline{}{Preface and acknowledgements}}
\chapter*{Preface and acknowledgements}
This document contains information on the Monte Carlo
X-ray tracing program \MCX version \version, building on the initial
release in October 1998 of the nuetron ray tracing program \MCX version 1.0 as presented in Ref.~\cite{nn_10_20}. The reader of this
document is expected to have some knowledge of X-ray and/or neutron scattering,
whereas only little knowledge about simulation techniques is
required. In a few places, we also assume familiarity with the
use of the C programming language and UNIX/Linux.

%It is a pleasure to thank Prof.~Kurt N.~Clausen, PSI, for his continuous
%support to this project and for having initiated \MCX\ in the first
%place. 
Support has also been given by SAXLAB Aps. through its CEO Karsten Joensen as well 
the ESRF and Max-LAB. We acknowledge all contributing parties. 


In case of errors, questions, or suggestions,
%or other need for support should aris,
do not hesitate to
contact the authors at \verb+@risoe.dk+
or consult the \MCX home page~\cite{mcxtrace_webpage}.
A special bug/request reporting service is available \cite{mczilla_webpage}.

If you {\bf appreciate} this software, please subscribe to the \verb+mcxtrace-users@mcxtrace.org+ email list, send us a smiley message, and contribute to the package. 

%We also encourage you to refer to this software when publishing results, with the following citations:
%\begin{itemize}
%\item{K. Lefmann and K. Nielsen, Neutron News {\bf 10/3}, 20, (1999).}
%\item{P. Willendrup, E. Farhi and K. Lefmann, Physica B, {\bf 350} (2004) 735.}
%\end{itemize}


%\section*{\MCX\ \version\ contributors}

Third party software included in \MCX\ is:
\begin{itemize}
\item perl Math::Amoeba from John A.R. Williams \verb+J.A.R.Williams@aston.ac.uk+.
\item perl Tk::Codetext from Hans Jeuken \verb+haje@toneel.demon.nl+.
\item scilab Plotlib from St\'ephane Mottelet \verb+mottelet@utc.fr+.
\item and optionally PGPLOT from Tim Pearson \verb+tjp@astro.caltech.edu+.
\end{itemize}

The \MCX\ project is supported by Det Strategiske Forskningsråd through the NaBiIT programme. Partners in this joint venture are: 
\begin{itemize}
\item Materials Research Division, Risø DTU, Roskilde, Denmark\\
\item Niels Bohr Institute, University of Copenhagen, Denmark\\
\item Faculty of Life Sciences, University of Copenhagen, Denmark\\
\item SAXLAB ApS. Lundtofte, Denmark\\
\item ESRF, Grenoble, France
\end{itemize}

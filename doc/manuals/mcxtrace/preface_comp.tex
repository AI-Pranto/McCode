% Emacs settings: -*-mode: latex; TeX-master: "manual.tex"; -*-

\addcontentsline{toc}{chapter}{\protect\numberline{}{Preface and acknowledgements}}
\chapter*{Preface and acknowledgements}
This document contains information on the x-ray scattering components
which are the building blocks for defining instruments
in the Monte Carlo Xray-tracing program \MCX version \version . The initial
release in June 2011 of version 1.0 was presented in Ref.~\cite{nn_10_20}.
The reader of this
document is not supposed to have specific knowledge of xray scattering,
but some basic understanding of physics is helpful in
understanding the theoretical background for the component functionality.
For details about setting up and running simulations, we refer to
the \MCX system manual \cite{mcxtracemanual}.
We assume familiarity with the use of
the C programming language.

%We especially like to thank Kristian Nielsen for laying a solid foundation
%for the \MCX\ system, which the authors of this manual benefit from daily.
We would like to explicitly thank all the partners in this project:
\begin{itemize}
\item The European Synchrotron Radiation Facility (ESRF), Grenoble, France
\item SAXSLAB Aps., Lundtofte, Denmark
\item Risø DTU, Roskilde, Denmark
\item \NBIlong
\item \Lifelong
\end{itemize}

As
\MCX has inherited much of its functionality from its sister \MCS we take the oppurtunity to thank 
Dir.~Kurt N.~Clausen, PSI, for his continuous
support to \MCS\ and for having initiated the project.
Continuous support to \MCS\ has also come from Prof.~Robert McGreevy, ISIS.
Apart from the authors of this manual, also Per-Olof \AA strand, NTNU Trondheim,
has contributed to the development of the \MCS\ system.

We have further benefited
from discussions with many other people in the scattering
community, too numerous to mention here.

The users who contributed components to this manual are acknowledged
as authors of the individual components. We encourage other
users to contribute components with manual entries for inclusion in
future versions of \MCX.

In case of errors, questions, or suggestions,
%or other need for support should arise,
do not hesitate to
contact the user/developer community by writing to the user mailiing list \verb+users@mcxtrace.órg+
or consult the \MCX\ home page~\cite{mcxtrace_webpage}. A special devlopement website (shared with the sister project \MCX) complete with bug/request reporting service is available \cite{trac_webpage}.

Some highlight of the feature in this the first release of McXtrace:


We would like to kindly thank all \MCX\ component contributors. This is the way we improve the software alltogether.


If you {\bfseries appreciate} this software, please subscribe to the \verb+users@mcxtrace.org+ email list, send us a smiley message, and contribute to the package.
%We also encourage you to refer to this software when publishing results, with the following citations:
%\begin{itemize}
%\item{K. Lefmann and K. Nielsen, Neutron News {\bfseries 10/3}, 20, (1999).}
%\item{P. Willendrup, E. Farhi and K. Lefmann, Physica B, {\bfseries 350}, 735 (2004).}
%\end{itemize}







\section{Virtual\_output: Saving the first part of a split simulation}
\label{s:virtual-output}
\index{Sources!Virtual source, recording photon events}
\mcdoccomp{misc/Virtual_output.parms}

The component \textbf{Virtual\_output} stores the photon ray parameters
at the end of the first part of a split simulation. The idea is to let the
next part of the split simulation be performed by another instrument file,
which reads the stored photon ray
parameters by the component \textbf{Virtual\_input}.

All photon ray parameters are saved to the output file, which is by default
of ``text'' \textit{type}, but can also assume the binary formats
``float'' or ``double''. The storing of photon rays continues until the
specified number of simulations have been performed.

\textit{buffer-size} may be used to limit the size of the output file, but
absolute intentities are then likely to be wrong.
Exept when using MPI, we recommend to use the default value of zero, saving all photon rays.
The size of the file is then controlled indirectly with the general \textit{ncounts} parameter.

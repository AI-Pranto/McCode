The software package McStas is a tool for carrying out Monte Carlo
ray-tracing simulations of neutron scattering instruments with high
complexity and precision. The simulations can compute all aspects of the
performance of instruments and can thus be used to optimize the use of
existing equipment, design new instrumentation, and carry out virtual
experiments for e.g. training, experimental planning or data analysis. McStas
is based on a unique design where an automatic compilation process
translates high-level textual instrument descriptions into efficient
ANSI-C code. This design makes it simple to set up typical simulations
and also gives essentially unlimited freedom to handle more unusual
cases.

This report constitutes the reference manual for McStas, and,
together with the manual for the McStas components, it
contains full documentation of all aspects of the program. It covers
the various ways to compile and run simulations, a description of the
meta-language used to define simulations, 
%a full description of all
%algorithms used to calculate the effects of the various optical
%components in instruments, 
and some example simulations performed with
the program.

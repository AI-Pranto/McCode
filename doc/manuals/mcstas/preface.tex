% Emacs settings: -*-mode: latex; TeX-master: "manual.tex"; -*-

\addcontentsline{toc}{chapter}{\protect\numberline{}{Preface and acknowledgements}}
\chapter*{Preface and acknowledgements}
This document contains information on the Monte Carlo neutron
ray-tracing program \MCS version \version, building on the initial
release in October 1998 of version 1.0 as presented in Ref.~\cite{nn_10_20}. The reader of this
document is supposed to have some knowledge of neutron scattering,
whereas only little knowledge about simulation techniques is
required. In a few places, we also assume familiarity with the
use of the C programming language and UNIX/Linux.

If you don't want to read this manual in full, go directly to the brief introduction in chapter \ref{s:brief}.

It is a pleasure to thank Prof.~Kurt N.~Clausen, PSI, for his continuous
support to this project and for having initiated \MCS in the first
place. Essential support has also been given by Prof.~Robert McGreevy, ISIS.
We have also benefited from discussions with many other people in the neutron scattering
community, too numerous to mention here.

In case of errors, questions, or suggestions, do not hesitate to
contact the authors at \verb+mcstas-support@mcstas.org+
or consult the \MCS home page~\cite{mcstas_webpage}.
A special bug/request reporting service is available \cite{mczilla_webpage}.

If you {\bf appreciate} this software, please subscribe to the \verb+mcstas-users@mcstas.org+ email list, send us a smiley message, and contribute to the package. We also encourage you to refer to this software when publishing results, with the following citations:
\begin{itemize}
\item{K. Lefmann and K. Nielsen, Neutron News {\bf 10/3}, 20, (1999).}
\item{P. Willendrup, E. Farhi and K. Lefmann, Physica B, {\bf 350} (2004) 735.}
\item{P. Willendrup, E. Farhi, E. Knudsen, U. Filges and K. Lefmann,
    Journal of Neutron Research, {\bf 17} (expected 2013)}
\end{itemize}


\section*{\MCS \version\ contributors}
Several people outside the core developer team have been contributing
to \MCS \version:
\begin{itemize}
\item UPDATE THIS LIST!!
\end{itemize}
Thank you guys! This is what \MCS is all about!

Third party software included in \MCS are:
\begin{itemize}
\item UPDATE THIS LIST!!!
\item perl Math::Amoeba from John A.R. Williams \verb+J.A.R.Williams@aston.ac.uk+.
\item perl Tk::Codetext from Hans Jeuken \verb+haje@toneel.demon.nl+.
\item scilab Plotlib from St\'ephane Mottelet \verb+mottelet@utc.fr+.
\item and optionally PGPLOT from Tim Pearson \verb+tjp@astro.caltech.edu+.
\end{itemize}
%and special thanks to the following \MCS contributors:
%\begin{itemize}
%\item Christophe Taton from ENSIMAG/Grenoble for the MPI support
%\item Reynald Arnerin from Universite J. Fourrier, Grenoble for the complex geometry support in samples
%\end{itemize} 

The \MCS project has been supported by the European Union
through``XENNI / Cool Neutrons'' (FP4), ``SCANS'' (FP5),
``nmi3/MCNSI'' (FP6). \MCS was supported directly from the construction project for the ISIS second
target station (TS2/EU), see~\cite{ts2_webpage}. Currently \MCS is 
supported through Danish \emph{in-kind} work packages toward the
European Spallation Source (ESS), see~\cite{ess_webpage} and the
European Union through ``nmi3/E-learning'' and ``nmi3/MCNSI7'' (FP7) - see
the home pages~\cite{nmi3_webpage,mcnsi_webpage}.


% Emacs settings: -*-mode: latex; TeX-master: "manual.tex"; -*-

\chapter{New features in \MCS \version\ }
\label{c:changes}

UPDATE!!!!

This version of \MCS implements both new features, as well as many bug corrections. Bugs are reported and traced using the \MCS Trac Ticket system \cite{mczilla_webpage}. We will not present here an extensive list of corrections, and we let the reader refer to this bug reporting service for details. Only important changes are indicated here.

Of course, we can not guarantee that the software is bullet proof, but we do our best to correct bugs, when they are reported.\index{Bugs}

%\section{General}
%\label{s:new-features:general}
%\begin{itemize}
%\end{itemize}

\section{Kernel}
\label{s:new-features:kernel}
\index{Kernel}

The following changes concern the 'Kernel' (i.e. the \MCS meta-language and program). See the dedicated chapter in the {\it User manual} for more details.

\begin{itemize}
\item The \verb+STATE PARAMETERS+ keywords has been removed from components code. We now assume that the neutron state parameters are \verb+x,y,z,vx,vy,vz,sx,sy,sz,t+. \MCS 1.X components will need to be adapted by commenting this line.
\item The \verb+PREVIOUS+ and \verb+MYSELF+ keywords can be used in component instance parameters and positioning (AT/ROTATED) so that one can make use of \verb+MC_GETPAR(PREVIOUS, parameter)+ directly in the TRACE section.
\end{itemize}

\section{Run-time}
\label{s:new-features:run-time}
\index{Library!Run-time}

\begin{itemize}
\item The number of neutron counts can now exceed $2.10^9$, being stored as a \verb+long long+.
\item The writing of files has been improved. The \verb+DETECTOR_OUT+ functions now return a \verb+mcdetector+ structure which holds all written information. Simulations also write a 'content.sim' file aside the 'mcstas.sim' one, which holds the integrated counts for each monitor.
\item A bug has been fixed in the rectangular focusing routine, for large solid angles. Reported by M. Skoulatos, L. Udby and J. Jacobsen.
\end{itemize}

\section{Components and Library}
\label{s:new-features:components}
\index{Components} \index{Library!Components}
We here list the new and updated components (found in the \MCS \verb+lib+ directory)
which are detailed in the {\it Component manual}, also mentioned in
the {\it Component Overview} of the {\it User Manual}. Generally, most component parameter names have been uniformized.

\subsection{Components}
\begin{itemize}
\item \verb+Single_crystal.comp+ now provides a lattice curvature option (M. Schulz, FRM2).
\item \verb+PowderN.comp+ now handles strain (E. Farhi, R. Rogge).
\item \verb+Monitor_nD+ was wrong in the determination of the flux per cm2 and steradians.
\item \verb+Guide_anyshape+ allows to model any reflecting geometry from a set of vertices and polygons.
\item \verb+Lens+ models refractive lenses and prisms (E. Farhi/C. Monzat, ILL).
\item \verb+Lens_simple+ for refractive lenses with a simple analytical model (H. Frielinghaus).
\item \verb+PSD_Detector+ can now work in event mode (E. Farhi).
\item \verb+Multilayer_Sample+ to model a set of refractive layers for e.g. reflectometry. Requires GNU Scientific Library to be previously installed (R. Dalgliesh, ISIS).
\item 
\end{itemize}

\subsection{New example instruments and Data files}
\begin{itemize}
\item Updated ILL instrument models: H16 guide, IN22, IN12, D16
\item New ILL instrument models: IN1, IN8, IN20, D2B, D4, IN5
\item Added HZB NEAT ToF spectrometer model (E. Farhi, R. Lechner)
\item Adde ISIS\_CRISP which uses new multilayer sample (R. Dalgliesh, ISIS)
\item Sort test instruments in various categories
\item Update of many data files.
\end{itemize}

\section{Documentation}
\label{s:new-features:documentation}
\begin{itemize}
\item Manual and component manual fully updated 
\end{itemize}

\section{Tools, installation}
\label{s:new-features:tools}
\index{Tools}
\index{Installing}
\subsection{New tool features}
\begin{itemize}
  \item Support for per-user mcstas\_config.perl file, located in \verb+$HOME/.mcstas/+ . This folder is also the default
     location of the 'host list' for use with MPI or gridding, simply name the file 'hosts'.
  \item mcgui Save Configuration for saving chosen settings on the 'Configuration options' and 'Run dialogue'.
  \item Possibilty to run MPI or grid simulations by default from mcgui.
  \item When scanning parameters, mcrun now terminates with a relevant error message if one or more scan steps
     failed (intensities explicitly set to 0 in those cases).
  \item When running parameter optimisations, a logfile (default name is "mcoptim\_XXXX.dat" where XXXX is a
     pseudo-random string) is created during the optimisation, updated at each optim step.
  \item We now provide syntax-highlighting setup files for vim and gedit editors.
  \item Rudimentary support for GNUPLOT when plotting with mcplot. Data file format is standard McStas/PGPLOT.
\end{itemize}
\subsection{Platform support}
\begin{itemize}
\item Mac OS X 10.3 Panther (ppc), 10.4 Tiger (pcc/intel), 10.5 Leopard (ppc/intel)
\item Windows XP,  Windows Vista (Now with a recent perl version; 5.10 plus various fixes). New feature on Windows:
     Simulations \emph{always} run in the background, freeing mcgui for other work.
\item "Any" Linux - reference platforms are Ubuntu 8.04 (and earlier) and Debian 4.0 (and earlier). We have also tested 
  Fedora 8, OpenSuSE 10.3 and CentOS 4 releases recently.
\item FreeBSD (FreeBSD release 6.3 and its cousin DesktopBSD 1.6 recently tested)
\item SUN Solaris 10 (Intel tested, Sparc probably OK)
\item Plus probably any UNIX/POSIX type environment with a bit of effort...
\end{itemize}
% Details about the installation and the available tools are given in chapter \ref{installing}.

\subsection{Various}
\begin{itemize}
\item  A number of minor bugs ironed out, both in components, runtime code and tools.
\item From release 1.12, McStas is GPL 2 only. The debate on the internet about the future GPL 3 license suggests that this license 
     might have implications on the 'derived work', hence have implications on what and how our users use their McStas simulations
     for. To protect user freedom, we will stick with GPL 2.     
\end{itemize}

\subsection{Warnings}
{\bf WARNING:} The 'dash' shell which is used as /bin/sh on some Linux system (Including Ubuntu 7.04) makes the 'Cancel' and 'Update' 
buttons fail in mcgui. Solutions are:
\begin{itemize}
\item[a)] If your system is a Debian or Ubuntu, please dpkg-reconfigure dash and say 'no' to install dash as /bin/sh
\item[b)] If you run another Linux with /bin/sh beeing dash, please install bash and manually change the /bin/sh link to point at bash.
\end{itemize}

\section{Future extensions}
\label{s:future}
The following features are planned for the oncoming releases of \MCS
(not an ordered list):
\begin{itemize}
\item Increased validation and testing.
\item Extend test cases to all (most) components. One instrument
  pr. component. (Probably not in \verb+examples/+.
\item Updates to mcresplot to support the Matlab backend.
\item Global changes of components relating to polarisation
  visualisation.
\item Visualisation of neutron spins in magnetic fields for all
  graphical backends.
\item \emph{Array} \verb+AT+ specifiers for components, i.e. \\
  \verb+COMPONENT MyComp=Comp(...)+\\\verb+AT([Xarray],[Yarray],[Zarray])+ and\\
  \verb+AT Positions('filename')+
\item Gui support for array AT specifiers.
\item More complete polarisation support including numerically defined
  magnetic fields and advanced sample components.
\item Perl or python plotting alternative to PGPLOT.
\item Larger variety of sample components.
\end{itemize}









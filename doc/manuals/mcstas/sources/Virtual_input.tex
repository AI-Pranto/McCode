\section{Virtual\_input: Starting the second part of a split simulation}
\label{virtual_input}
\index{Sources!Virtual source from stored neutron events}

%\component{Virtual\_input}{System}{filename}{repeat-count, type}{}
\mcdoccomp{sources/Virtual_input.parms}

The component \textbf{Virtual\_input} resumes a split simulation where the
first part has been performed by another instrument and the neutron ray
parameters have been stored by the component \textbf{Virtual\_output}.

All neutron ray parameters are read from the input file, which is by default
of ``text'' type, but can also assume the binary formats
``float'' and ``double''. The reading of neutron rays continues until the
specified number of rays have been simulated or
till the file has been exhausted. If desirable, the input file
can be reused a number of times, determined by the optional parameter
``repeat-count''. This is only useful if the present simulation makes use of
MC choices, otherwise the same outcome will result for each repetition of the
simulation (see Appendix \ref{s:MCtechniques}).

Care should be taken when dealing with
absolute intensities, which will be correct only
when the input file has been exhausted at least once.

The simulation ends with either the end of the repeated file counts,
or with the normal end with $ncount$ \MCS simulation events. We recommend to
control the simulation on \verb+repeat-count+ by using
a very larger ncount value.

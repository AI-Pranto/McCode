\section{Source\_Optimizer: A general Optimizer for McStas}
\label{source-optimizer}
\index{Sources!Optimizer}\index{Optimization}
\component{Source\_Optimizer}{E. Farhi, ILL}{options}{bins, step, keep}{partially validated}

The component {\bf Source\_Optimizer} is not exactly a source,
but rather a neutron beam modifier.
It should be positioned after the source, anywhere in the instrument description.
The component  optimizes the whole neutron flux
in order to achieve better statistics at each {\bf Monitor\_Optimizer}
location(s) (see section~\ref{monitor-optimizer} for this latter
component). It acts on any incoming neutron beam (from any source
type), and more than one optimization criteria location can be placed
along the instrument.

The usage of the optimizer is very simple, and usually does not require
any configuration parameter. Anyway the user can still customize the
optimization through various {\it options}.

In contrast to {\bf Source\_adapt}, this optimizer does not
record correlations between neutron parameters.
Nevertheless it is rather efficient,
enabling the user to increase the number of events
at optimization criteria locations by typically a factor of 20.
Hence, the signal error bars will decrease by a factor 4.5,
since the overall flux remains unchanged.

\subsection{The optimization algorithm}

When a neutron reaches the {\bf Monitor\_Optimizer} location(s), the
component records its previous position ($x$, $y$) and speed ($v_x,
v_y, v_z$) when it passed in the {\bf Source\_Optimizer}. Some
distribution tables of {\it good} neutrons characteristics are then
built.

When a {\it bad} neutron comes to the {\bf Source\_Optimizer} (it would
then have few chances to reach {\bf Monitor\_Optimizer}), it is changed
into a better one. That means that its position and velocity coordinates
are translated to better values according to the {\it good} neutrons
distribution tables. The neutron energy
($\sqrt{v_x^2 + v_y^2 + v_z^2}$) is kept (as far as possible).

The {\bf Source\_Optimizer} works as follow:
\begin{enumerate}
\item{First of all, the {\bf Source\_Optimizer} determines some limits
    ({\it min} and {\it max}) for variables $x, y, v_x, v_y, v_z$.}
\item{Then the component records the non-optimized flux distributions in
    arrays with {\it bins} cells (default is 10 cells). This constitutes
    the {\it Reference } source.}
\item{\label {SourceOptimizer:step3}The {\bf Monitor\_Optimizer} records
    the {\it good} neutrons (that reach it) and communicate an {\it
      Optimized} beam requirement to the {\bf Source\_Optimizer}. However, retains '{\it
      keep}' percent of the original {\it Reference} source is sent
    unmodified (default is 10 \%). The {\it Optimized} source is thus:

    \begin{center}
      \begin{tabular}{rcl}
        {\it Optimized} & = & {\it keep} * {\it Reference} \\
        & + & (1 - {\it keep}) [Neutrons that will reach monitor].
      \end{tabular}
    \end{center}
    }
\item{The {\bf Source\_Optimizer} transforms the {\it bad} neutrons into
    {\it good} ones from the {\it Optimized} source. The resulting
    optimised flux is normalised to the non-optimized one:
    \begin{equation}
      p_{optimized} = p_{initial} \frac{\mbox{Reference}}{\mbox{Optimized}},
    \end{equation}
    and thus the overall flux at {\bf Monitor\_Optimizer} location is
    the same as without the optimizer. Usually, the process sends more
    {\it good} neutrons from the {\it Optimized} source than that in the
    {\it Reference} one.
    The energy (and velocity) spectra of neutron beam is also kept, as
    far as possible. For instance, an optimization of $v_z$ will induce
    a modification of $v_x$ or $v_y$ to try to keep $|{\bf v}|$
    constant.
    }
\item{When the {\it continuous} optimization option is activated (by
    default), the process loops to Step (\ref{SourceOptimizer:step3})
    every '{\it step}' percent of the simulation. This parameter is
    computed automatically (usually around 10 \%) in {\it auto} mode,
    but can also be set by user.}
\end{enumerate}

During steps (1) and (2), some non-optimized neutrons with original
weight $p_{initial}$ may lead to spikes on detector signals. This is
greatly improved by lowering the weight $p$ during these steps, with the
{\it smooth} option.
The component optimizes the neutron parameters on the basis of
independant variables (1D phase-space optimization). However, it usually does work fine when these
variables are correlated (which is often the case in the course of the
instrument simulation).
The memory requirements of the component are very low, as no big
$n$-dimensional array is needed.

\subsection{Using the Source\_Optimizer}

To use this component, just install the {\bf Source\_Optimizer} after a
source (but any location is possible afterwards in principle), and use the {\bf
  Monitor\_Optimizer} at a location where you want to reach better
statistics.

\begin{lstlisting}
    /* where to act on neutron beam */
    COMPONENT optim_s = Source_Optimizer(options="")
    ...
    /* where to have better statistics */
    COMPONENT optim_m = Monitor_Optimizer(
    xmin = -0.05, xmax = 0.05,
    ymin = -0.05, ymax = 0.05,
    optim_comp = optim_s)
    ...
    /* using more than one Monitor_Optimizer is possible */
\end{lstlisting}

The input parameter for {\bf Source\_Optimizer} is a single {\it
  options} string that can contain some specific optimizer configuration
settings in clear language. The formatting of the {\it options}
parameter is free, as long as it contains some specific keywords, that
can be sometimes followed by values.

The default configuration (equivalent to {\it options} = "") is
\begin{center}
\begin{tabular}{rcl}
  {\it options} & = & "{\it continuous} optimization,
  {\it auto} setting, {\it keep} = 0.1, {\it bins} = 0.1, \\
  & & {\it smooth} spikes, SetXY+SetDivV+SetDivS".
\end{tabular}
\end{center}
Parameters keep and step should be between 0 and 1.
Additionally, you may restrict the optimization to only some of the neutron parameters, using the {\it SetXY, SetV, SetS, SetDivV, SetDivS} keywords.
The keyword modifiers {\it no} or {\it not} revert the next option.
Other options not shown here are:
\begin{lstlisting}
verbose         displays optimization process (debug purpose).
unactivate      to unactivate the Optimizer.
file=[name]     Filename where to save optimized source distributions
\end{lstlisting}
The {\it file} option will save the source distributions at the end of
the optimization. If no name is given the component name will be used,
and a '.src' extension will be added. By default, no file is generated.
The file format is in a McStas 2D record style.

As an alternative, you may use the Source\_adapt component
(see section \ref{s:source-adapt}) which performs
a 3D phase-space optimization.

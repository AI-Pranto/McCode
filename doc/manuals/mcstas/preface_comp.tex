% Emacs settings: -*-mode: latex; TeX-master: "manual.tex"; -*-

\addcontentsline{toc}{chapter}{\protect\numberline{}{Preface and acknowledgements}}
\chapter*{Preface and acknowledgements}
This document contains information on the neutron scattering components
which are the building blocks for defining instruments
in the Monte Carlo neutron
ray-tracing program \MCS\ version \version . The initial
release in October 1998 of version 1.0 was presented in Ref.~\cite{nn_10_20}.
The reader of this
document is not supposed to have specific knowledge of neutron scattering,
but some basic understanding of physics is helpful in
understanding the theoretical background for the component functionality.
For details about setting up and running simulations, we refer to
the \MCS\ system manual \cite{mcstasmanual}.
We assume familiarity with the use of
the C programming language.

%We especially like to thank Kristian Nielsen for laying a solid foundation
%for the \MCS\ system, which the authors of this manual benefit from daily.

It is a pleasure to thank Dir.~Kurt N.~Clausen, PSI, for his continuous
support to \MCS\ and for having initiated the project.
Continuous support to \MCS\ has also come from Prof.~Robert McGreevy, ISIS.
Apart from the authors of this manual, also Per-Olof \AA strand, NTNU Trondheim,
has contributed to the development of the \MCS\ system.
%Both he and our other collaborators, Henrik M.\ R\o nnow and Mark
%Hagen have made major contributions to the project.  Also the
%contributions from our test users, the students Asger Abrahamsen, Niels
%Bech Christensen, and Erik Lauridsen, are gratefully acknowledged; they
%gave us an excellent opportunity to pinpoint a vast amount of serious
%errors in the test version.  Useful comments to this document itself
%have been given by Bella Lake and Alan Tennant.
We have further benefited
from discussions with many other people in the neutron scattering
community, too numerous to mention here.

%Philipp Bernhardt contributed the two chopper components in
%sections~\ref{s:chopper} and~\ref{s:first_chopper}. Emmanuel Farhi
%contributed the components in sections~\ref{s:sourceoptimizer},
%\ref{s:monitornd}, and~\ref{s:monitoroptimizer}.
The users who contributed components to this manual are acknowledged
as authors of the individual components. We encourage other
users to contribute components with manual entries for inclusion in
future versions of \MCS.

In case of errors, questions, or suggestions,
%or other need for support should arise,
do not hesitate to
contact the authors at \verb+mcstas@risoe.dk+
or consult the \MCS\ home page~\cite{mcstas_webpage}. A special bug/request reporting service is available \cite{mczilla_webpage}.

Important developments on the component side in \MCS\ version \version\
as compared to version 1.4 (the last version of the component manual;
then a section of the system manual) include

\begin{itemize}
\item{Validation of most components against analytical formula,
and benchmarking in simple cases}
\item{Newly added, realistic source components}
  \begin{itemize}
  \item \verb+ISIS_moderator+ ISIS source model based on MCNPX (D. Champion and S. Ansell, ISIS)
  \item \verb+Virtual_tripoli4_input/output+ Trioli4 (similar to MCNP) files reading/writing (G. Campioni, LLB)
  \item \verb+SNS_source+ SNS source model based on MCNPX (G. Granroth, SNS)
  \item \verb+Source_gen+ ILL sources Maxwellian parameters (E. Farhi/N. Kernavanois/H. Bordallo, ILL)
  \item \verb+ESS_moderator_short+ Calculated source model for the short pulse target station of the ESS project (K. Lefmann, Ris\o )
  \item \verb+ESS_moderator_long+ Calculated source model for the long pulse target station of the ESS project (K. Lefmann, Ris\o )
  \end{itemize}
\item{Newly added, optical components}
  \begin{itemize}
  \item \verb+Radial_collimator+ Radial collimator with both approximated and exact options (E. Farhi, ILL)
  \item \verb+FermiChopper+ and \verb+Vitess_ChopperFermi+ Two Fermi Chopper components (M. Poehlmann, G. Zsigmond, ILL and PSI)
  \item \verb+Guide_tapering+ A rectangular tapered guide (U. Filges, PSI)
  \item \verb+Guide_curved+  Non-focusing curved neutron guide (R. Stewart, ILL)
  \end{itemize}
\item{A suite of sample components}
  \begin{itemize}
  \item \verb+Inelastic_Incoherent+ Inelastic incoherent sample with quasielastic and elastic contributions (K. Lefmann, Ris\o)
  \item \verb+Phonon_simple+ An isotropic acoustic phonon (K. Lefmann, Ris\o)
  \item \verb+PowderN+. N lines powder diffraction (P.K. Willendrup, Ris\o)
  \item \verb+Sans_spheres+ hard spheres in thin solution, mono disperse (L. Arleth, the Royal Veterinary and Agricultural University (DK), K. Lefmann, Ris\o )
  \item \verb+Isotropic_Sqw+ isotropic inelastic sample (powder, liquid, glass)
elastic/inelastic scattering from $S(q,\omega)$ data (E. Farhi, V. Hugouvieux, ILL)
  \item \verb+SANS_*+ A collection of samples for SANS (H. Frielinghaus,  FZ-J\"ulich)
  \end{itemize}
\end{itemize}

We would like to kindly thank all \MCS\ component contributors. This is the way we improve the software alltogether.

The \MCS\ project has been supported by the European Union, initially
through the XENNI program and the RTD ``Cool Neutrons'' program in FP4,
In FP5, \MCS\ was supported strongly through the
``SCANS'' program.
Currently, in FP6, \MCS\ is supported through the Joint Research Activity
``MCNSI'' under the Integrated Infrastructure Initiative ``NMI3'', see
the WWW home pages~\cite{mcnsi_webpage,nmi3_webpage}.

If you {\bf appreciate} this software, please subscribe to the \verb+neutron-mc@risoe.dk+ email list, send us a smiley message, and contribute to the package. We also encourage you to refer to this software when publishing results, with the following citations:
\begin{itemize}
\item{K. Lefmann and K. Nielsen, Neutron News {\bf 10/3}, 20, (1999).}
\item{P. Willendrup, E. Farhi and K. Lefmann, Physica B, {\bf 350}, 735 (2004).}
\end{itemize}







% Emacs settings: -*-mode: latex; TeX-master: "manual.tex"; -*-

\addcontentsline{toc}{chapter}{\protect\numberline{}{Preface and acknowledgements}}
\chapter*{Preface and acknowledgements}
This document contains information on the neutron scattering components
which are the building blocks for defining instruments
in the Monte Carlo neutron
ray-tracing program \MCS version \version . The initial
release in October 1998 of version 1.0 was presented in Ref.~\cite{nn_10_20} and further developed though version 2.0 as
presented in Ref. ~\cite{mcs_ppf}.
The reader of this
document is not supposed to have specific knowledge of neutron scattering,
but some basic understanding of physics is helpful in
understanding the theoretical background for the component functionality.
For details about setting up and running simulations, we refer to
the \MCS system manual \cite{mcstasmanual}.
We assume familiarity with the use of
the C programming language.

%We especially like to thank Kristian Nielsen for laying a solid foundation
%for the \MCS system, which the authors of this manual benefit from daily.

It is a pleasure to thank Dir.~Kurt N.~Clausen, PSI, for his continuous
support to \MCS and for having initiated the project.
Continuous support to \MCS has also come from Prof.~Robert McGreevy, ISIS.
Apart from the authors of this manual, also Per-Olof \AA strand, NTNU Trondheim,
has contributed to the development of the \MCS system.
%Both he and our other collaborators, Henrik M.\ R\o nnow and Mark
%Hagen have made major contributions to the project.  Also the
%contributions from our test users, the students Asger Abrahamsen, Niels
%Bech Christensen, and Erik Lauridsen, are gratefully acknowledged; they
%gave us an excellent opportunity to pinpoint a vast amount of serious
%errors in the test version.  Useful comments to this document itself
%have been given by Bella Lake and Alan Tennant.
We have further benefited
from discussions with many other people in the neutron scattering
community, too numerous to mention here.

%Philipp Bernhardt contributed the two chopper components in
%sections~\ref{s:chopper} and~\ref{s:first_chopper}. Emmanuel Farhi
%contributed the components in sections~\ref{s:sourceoptimizer},
%\ref{s:monitornd}, and~\ref{s:monitoroptimizer}.
The users who contributed components to this manual are acknowledged
as authors of the individual components. We encourage other
users to contribute components with manual entries for inclusion in
future versions of \MCS.

In case of errors, questions, or suggestions,
%or other need for support should arise,
do not hesitate to
contact the authors at \verb+mcstas@risoe.dk+
or consult the \MCS home page~\cite{mcstas_webpage}. A special bug/request reporting service is available \cite{github_issue_webpage}.

We would like to kindly thank all \MCS component contributors. This is the way we improve the software alltogether.

The \MCS project has been supported by the European Union
through``XENNI / Cool Neutrons'' (FP4), ``SCANS'' (FP5),
``nmi3/MCNSI'' (FP6), ```nmi3-ii/E-learning'' and ``nmi3-ii/MCNSI7''
(FP7) ~\cite{nmi3_webpage,mcnsi_webpage}.
\MCS was supported directly from the construction project for the ISIS second
target station (TS2/EU), see~\cite{ts2_webpage}. Currently \MCS is 
supported through the Danish involvement in the \emph{Data Management
  and Software Center}, a subdivision of the 
European Spallation Source (ESS), see~\cite{ess_webpage} and the
European Union through ``SINE2020/WP3 e-elarning''
and and ``SINE2020/WP8 e-Tools'' (Horizon2020).
the home pages~\cite{sine2020_webpage}.


If you \textbf{appreciate} this software, please subscribe to the \verb+neutron-mc@risoe.dk+ email list, send us a smiley message, and contribute to the package. We also encourage you to refer to this software when publishing results, with the following citations:
\begin{itemize}
\item{P. Willendrup, E. Farhi, E. Knudsen, U. Filges and K. Lefmann,
    Journal of Neutron Research, \textbf{17} (2014) 35.}
\item{K. Lefmann and K. Nielsen, Neutron News \textbf{10/3}, 20, (1999).}
\item{P. Willendrup, E. Farhi and K. Lefmann, Physica B, \textbf{350}
    (2004) 735.}
\end{itemize}






